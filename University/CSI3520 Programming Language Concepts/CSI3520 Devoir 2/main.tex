
%% ==================================================================================================
%%
\documentclass[12pt]{book}
\usepackage{amsfonts}
\usepackage{amsmath}
\usepackage{amssymb}
\usepackage{graphicx}
\usepackage{hyperref}
\usepackage{float}
\usepackage{verbatim}
\usepackage{xlop} %% for multiplication https://tex.stackexchange.com/questions/11702/how-to-present-a-vertical-multiplication-addition
\usepackage{listings} %% to format generic computer code
\usepackage{lmodern} % for bold teletype font
\usepackage{minted} % colour Java code

\usepackage{tasks}
%\NewTasks[style=enumerate,counter-format=tsk[A].,label-width=3ex]{choice}[\item](4)

%% =======   set page margins    =======
\setlength{\textheight}{10in}
\setlength{\textwidth}{7.4in}
\setlength{\topmargin}{-0.75in}
\setlength{\oddsidemargin}{-0.5in}
\setlength{\evensidemargin}{-0.5in}
\setlength{\parskip}{0.15in}
\setlength{\parindent}{0in}

%%  for European long division
% https://tex.stackexchange.com/questions/432435/how-to-set-up-european-french-style-long-division-in-tex
\newcommand\frdiv[5]{%
    \[
    \renewcommand\arraystretch{1.5}
    \begin{array}{l| l}
    #1 & #2 \\
    \cline{2-2}
    #3 & #4 \\
    \cline{1-1}
    #5 & \\
    \end{array}
    \]
}

%%  for European long division


%% ==================================================================================================

\begin{document}

%\title{ITI1100 Digital Systems I}
%\author{Kien Do 300163370}
%\date{Assignment \#1}
\newcommand{\reporttitle}{Devoir 2}
\newcommand{\reportauthorOne}{Kien Do}
\newcommand{\cidOne}{300163370}
\input{titlePage/titlepage.txt}



%% ==================================================================================================

%%%%%%%%%%%% PROBLEMS START HERE

\begin{enumerate}
    \item \textbf{Réponse}\\
    
    Selon les règles de portée \textbf{statique}, la valeur de $x$ affichée dans la fonction sub1() est 5.\\
    Selon les règles de portée \textbf{dynamique}, la valeur de $x$ affichée dans la fonction sub1() est 10.
    \item \textbf{Réponse}
\begin{minted}[breaklines,frame=single]{java}
Variable                 Où déclaré

Dans sub1():
    a                    sub1
    y                    sub1
    z                    sub1
    x                    main
    
Dans sub2():
    a                    sub2
    b                    sub2
    z                    sub2
    x                    main
    y                    main

Dans sub3():
    a                    sub3
    x                    sub3
    w                    sub3
    y                    main
    z                    main
\end{minted}
    \item \textbf{Réponse}
\begin{minted}[breaklines,frame=single]{java}
Variable                 Définition

Dans point 1:
    a                    1
    b                    2
    c                    2
    d                    2
Dans point 2:
    a                    1
    b                    2
    c                    3
    d                    3
    e                    3
Dans point 3:
    Même chose que point 1 puisque c'est le même scope.
Dans point 4:
    a                    1
    b                    1
    c                    1
\end{minted}

\newpage

    \item \textbf{Réponse}
    \begin{enumerate}
        \item \texttt{main} calls \texttt{fun1}; \texttt{fun1} calls \texttt{fun2}; \texttt{fun2} calls \texttt{fun3}.
\begin{minted}[breaklines,frame=single]{java}
Variable                 Où déclaré

d,e,f                    fun3
c                        fun2
b                        fun1
a                        main
\end{minted}
        \item \texttt{main} calls \texttt{fun1}; \texttt{fun1} calls \texttt{fun3}.
\begin{minted}[breaklines,frame=single]{java}
Variable                 Où déclaré

d,e,f                    fun3
b,c                      fun1
a                        main
\end{minted}
        \item \texttt{main} calls \texttt{fun2}; \texttt{fun2} calls \texttt{fun3}; \texttt{fun3} calls \texttt{fun1}.
\begin{minted}[breaklines,frame=single]{java}
Variable                 Où déclaré

b,c,d                    fun1
e,f                      fun3
a                        main
\end{minted}
        \item \texttt{main} calls \texttt{fun3}; \texttt{fun3} calls \texttt{fun1}.
\begin{minted}[breaklines,frame=single]{java}
Variable                 Où déclaré

b,c,d                    fun1
e,f                      fun3
a                        main
\end{minted}
        \item \texttt{main} calls \texttt{fun1}; \texttt{fun1} calls \texttt{fun3}; \texttt{fun3} calls \texttt{fun2}.
\begin{minted}[breaklines,frame=single]{java}
Variable                 Où déclaré

c,d,e                    fun2
f                        fun3
b                        fun1
a                        main
\end{minted}
        \item \texttt{main} calls \texttt{fun3}; \texttt{fun3} calls \texttt{fun2}; \texttt{fun2} calls \texttt{fun1}.
\begin{minted}[breaklines,frame=single]{java}
Variable                 Où déclaré

b,c,d                    fun1
e                        fun2
f                        fun3
a                        main
\end{minted}
    \end{enumerate}

\newpage    

    \item \textbf{Réponse}
    \begin{enumerate}
        \item \texttt{A -> aB | b | cBB}\\
        
        FIRST(aB) = a\\
        FIRST(b) = b\\
        FIRST(cBB) = c\\
        
        $a \cap b \cap c = \O$\\
        
        Le test de disjonction par paires réussit.
        
        \item \texttt{A -> aB | bA | aBb}\\
        
        FIRST(aB) = a\\
        FIRST(bA) = b\\
        FIRST(aBb) = a\\
        
        Le test de disjonction par paires échoue.
        
        \item \texttt{A -> aaA | b | caB}\\
        
        FIRST(aaA) = a\\
        FIRST(b) = b\\
        FIRST(caB) = c\\
        
        Le test de disjonction par paires réussit.
    \end{enumerate}
    
    \item \textbf{Réponse}\\
    
    Grammaire:
\begin{minted}[breaklines,frame=none]{java}
    S --> aAb | bBA
    A --> ab | aAB
    B --> aB | b
\end{minted}
        \begin{enumerate}
            \item \texttt{aaAbb}\\
            
            Arbre d'analyse (parse tree):
\begin{minted}[breaklines,frame=none]{java}
              S
            / | \
           a  A  b
            / | \
           a  A  B
                 |
                 b
\end{minted}
            Phrase(s): aaAbb, aaABb, aAb\\
            Phrase(s) simple: b\\
            Poingée (handle): b, aAB\\
            
            \item \texttt{bBab}\\
            
            Arbre d'analyse (parse tree):
\begin{minted}[breaklines,frame=none]{java}
              S
            / | \
           b  B  A
                / \
               a   b
\end{minted}
            Phrase(s): bBab, bBA\\
            Phrase(s) simple: ab\\
            Poingée (handle): ab\\
        \end{enumerate}


\end{enumerate}






\end{document} 


%% ==================================================================================================
%%
\documentclass[12pt]{book}
\usepackage{amsfonts}
\usepackage{amsmath}
\usepackage{amssymb}
\usepackage{graphicx}
\usepackage{hyperref}
\usepackage{float}
\usepackage{verbatim}
\usepackage{xlop} %% for multiplication https://tex.stackexchange.com/questions/11702/how-to-present-a-vertical-multiplication-addition
\usepackage{listings} %% to format generic computer code
\usepackage{lmodern} % for bold teletype font
\usepackage{minted} % colour Java code

\usepackage{tasks}
%\NewTasks[style=enumerate,counter-format=tsk[A].,label-width=3ex]{choice}[\item](4)

%% =======   set page margins    =======
\setlength{\textheight}{10in}
\setlength{\textwidth}{7.4in}
\setlength{\topmargin}{-0.75in}
\setlength{\oddsidemargin}{-0.5in}
\setlength{\evensidemargin}{-0.5in}
\setlength{\parskip}{0.15in}
\setlength{\parindent}{0in}

%%  for European long division
% https://tex.stackexchange.com/questions/432435/how-to-set-up-european-french-style-long-division-in-tex
\newcommand\frdiv[5]{%
    \[
    \renewcommand\arraystretch{1.5}
    \begin{array}{l| l}
    #1 & #2 \\
    \cline{2-2}
    #3 & #4 \\
    \cline{1-1}
    #5 & \\
    \end{array}
    \]
}

%%  for European long division


%% ==================================================================================================

\begin{document}

%\title{ITI1100 Digital Systems I}
%\author{Kien Do 300163370}
%\date{Assignment \#1}
\newcommand{\reporttitle}{Devoir 3}
\newcommand{\reportauthorOne}{Kien Do}
\newcommand{\cidOne}{300163370}
\input{titlePage/titlepage.txt}



%% ==================================================================================================

%%%%%%%%%%%% PROBLEMS START HERE

\begin{enumerate}
    % \item \textbf{Réponse}\\
    \item \textbf{Réponse}
\begin{minted}[breaklines,frame=single]{r}
# Create a vector `names` that contains your name and the names of 2 people
# next to you. Print the vector.
# Use the colon operator : to create a vector `n` of numbers from 10:49
# Use the `length()` function to get the number of elements in `n`
# Add 1 to each element in `n` and print the result
# Create a vector `m` that contains the numbers 10 to 1 (in that order).
# Hint: use the `seq()` function
# Subtract `m` FROM `n`. Note the recycling!
# Use the `seq()` function to produce a range of numbers from -5 to 10 in `0.1`
# increments. Store it in a variable `x_range`

names <- c("Kien", "Evan", "Shivan");
print(names);

n <- c(10:49);
print(length(n));
n <- n + 1;
print(n);


m <- seq.int(10, 1); # peut être c(10:1) également
print(m)

print(n - m);

x_range <- seq(-5, 10, 0.1);
print(x_range);
\end{minted}

% ===================================================================
\newpage
    \item \textbf{Réponse}
    
    \underline{Code en C}
\begin{minted}[breaklines,frame=single]{c}
#include <stdio.h>

int main()
{
    int j = 0;
    int k = (j + 13) / 27;
    
    while (k <= 10)
    {
        k = k + 1;
        int i = 3 * k - 1;
    }

    return 0;
}
\end{minted}
    \underline{Code en Java}
\begin{minted}[breaklines,frame=single]{java}
public class Main
{
    public static void main(String[] args) {
        int j = 0;
        int k = (j + 13) / 27;
        
        while (k <= 10) {
            k = k + 1;
            int i = 3 * k - 1;
        }
    }
}
\end{minted}
    \underline{Code en Python}
\begin{minted}[breaklines,frame=single]{python}
j = 0
k = (j + 13) / 27

while k <= 10:
    k = k + 1
    i = 3 * k - 1
\end{minted}

\underline{Meilleure écriture}: Python\\
--- Très facile et court à écrire. Pas besoin de spécifier les types des données.\\
\underline{Meilleure lisibilité}: C\\
--- Très facile à lire. Les accolades de chaque bloque de code commencent et terminent sur la même ligne verticale. Les types des données sont spécifiés, ce qui améliore la lisibilité.\\
\underline{Meilleure combinaison}: Java\\
--- Plus facile à lire par rapport à Python et plus facile à écrire par rapport à C.

% ===================================================================
\newpage
    \item \textbf{Réponse}\\

    Les valeurs de sum1 et sum2 si les opérandes dans les expressions sont évalués de:
    \begin{enumerate}
        \item \underline{Gauche à droite}\\

        sum1 = 46\\
        sum2 = 48\\
        
        \item \underline{Droite à gauche}\\

        sum1 = 48\\
        sum2 = 46\\
    \end{enumerate}

% ===================================================================
\newpage
    \item \textbf{Réponse}\\
    \underline{Le programme de l'exercice 3 en C++}
\begin{minted}[breaklines,frame=single]{c++}
#include <iostream>

using namespace std;

int fun(int *k)
{
    *k += 4;
    return 3 * (*k) - 1;
}

int main()
{
    int i = 10, j = 10, sum1, sum2;
    sum1 = (i / 2) + fun(&i);
    sum2 = fun(&j) + (j / 2);
    
    return 0;
}

// Résultats:
// sum1 = 46 et sum2 = 48
\end{minted}
\underline{Le programme de l'exercice 3 en Java}
\begin{minted}[breaklines,frame=single]{java}
public class Main
{
    public static int fun(int k) {
        k += 4;
        return 3 * k - 1;
    }
    
    public static void main(String[] args) {
        int i = 10, j = 10, sum1, sum2;
        sum1 = (i / 2) + fun(i);
        sum2 = fun(j) + (j / 2);
    }
}

// Résultats:
// sum1 = 46 et sum2 = 46 
\end{minted}
--- Les deux versions sont différentes parce que en Java, les entiers sont passés par valeur, et non passés par addresse, et vice versa. C'est-à-dire, en C++, les entiers i et j changent en main() car ils étaient changés en fun(). Ça fonctionne parce que ces deux entiers étaient passés par référence. Par contre, en Java, les entiers i et j ne changent pas en main() même s'ils étaient changés en fun(), parce que ces entiers sont passés par valeur.


% ===================================================================
\newpage
    \item \textbf{Réponse}\\
    
    Écrivez un programme en Java, C++, Python, qui effectue un grand nombre d'opérations en virgule flottante et un nombre égal d'opérations sur des nombres entiers et comparez le temps requis.\\

\underline{Java}
\begin{minted}[breaklines,frame=single]{java}
public class Main {
    public static void main(String[] args) {
        float total = 0;
        int count = 100000000;
        for (int i = 0; i < count; i++) { 
            total += 0.1; 
        } 
    }
}
\end{minted}

\underline{C++}
\begin{minted}[breaklines,frame=single]{c++}
#include <iostream>

using namespace std;

int main()
{
    float total = 0.0;
    int count = 100000000;
    for (int i = 0; i < count; i++)
    { 
        total += 0.1; 
    } 

    return 0;
}
\end{minted}

\underline{Python}
\begin{minted}[breaklines,frame=single]{python}
total = 0.0
count = 100000000
for x in range(count): 
    total += 0.1
\end{minted}

\underline{Java}: $\sim$3 secondes\\
--- Vitesse moyenne car Java est compilé et interprété.\\
\underline{C++}: $\sim$1 seconde\\
--- Vitesse la plus rapide car C++ est compilé et est très bon pour les calculs mathématiques.\\
\underline{Python}: $\sim$9 secondes\\
--- Vitesse la plus lente car Python est interprété.






\end{enumerate}

\end{document} 

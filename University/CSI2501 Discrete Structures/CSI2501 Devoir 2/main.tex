
%% ==================================================================================================
%%
\documentclass[12pt]{book}
\usepackage{amsfonts}
\usepackage{amsmath}
\usepackage{amssymb}
\usepackage{graphicx}
\usepackage{hyperref}
\usepackage{float}
\usepackage{verbatim}
\usepackage{xlop} %% for multiplication https://tex.stackexchange.com/questions/11702/how-to-present-a-vertical-multiplication-addition
\usepackage{listings} %% to format generic computer code
\usepackage{lmodern} % for bold teletype font
\usepackage{minted} % colour Java code

\usepackage{tasks}
%\NewTasks[style=enumerate,counter-format=tsk[A].,label-width=3ex]{choice}[\item](4)

%% =======   set page margins    =======
\setlength{\textheight}{10in}
\setlength{\textwidth}{7.4in}
\setlength{\topmargin}{-0.75in}
\setlength{\oddsidemargin}{-0.5in}
\setlength{\evensidemargin}{-0.5in}
\setlength{\parskip}{0.15in}
\setlength{\parindent}{0in}

%%  for European long division
% https://tex.stackexchange.com/questions/432435/how-to-set-up-european-french-style-long-division-in-tex
\newcommand\frdiv[5]{%
    \[
    \renewcommand\arraystretch{1.5}
    \begin{array}{l| l}
    #1 & #2 \\
    \cline{2-2}
    #3 & #4 \\
    \cline{1-1}
    #5 & \\
    \end{array}
    \]
}

%%  for European long division


%% ==================================================================================================

\begin{document}

\newcommand{\reporttitle}{Devoir 2}
\newcommand{\reportauthorOne}{Kien Do}
\newcommand{\cidOne}{300163370}
\input{titlePage/titlepage.txt}



%% ==================================================================================================

%%%%%%%%%%%% PROBLEMS START HERE

\begin{enumerate}
    \item Vrai
    
    Preuve par induction.
    
    Supposons que $\{x \in \mathbb{R} \,|\, x \geq -1\}$ et $\{n \in \mathbb{Z} \,|\, n \geq 2\}$. On veut prouver que 
    $$P(n) = \forall x \, \forall n \, ((1+x)^n \geq 1+nx)$$
    
    \textbf{Cas de base:}\\
    Soit $n = 2$. Montrons que $P(2)$ est vrai. On a que,
    $$(1+x)^n = (1+x)^2 = 1 + 2x + x^2$$
    Et,
    $$1 + nx = 1 + (2)x = 1 + 2x$$
    Donc, $P(n)$ est vrai pour $n = 2$.\\
    
    \textbf{Hypothèse d'induction}
    
    Supposons que $P(k) = \forall x \, \forall k \, ((1+x)^k \geq 1+kx)$ où $\{x \in \mathbb{R} \,|\, x \geq -1\}$ et $\{k \in \mathbb{Z} \,|\, k \geq 2\}$.\\
    
    \textbf{Étape d'induction}
    
    Il faut montrer que l'étape d'induction est vrai pour $P(k+1) = \forall x \, \forall k \, ((1+x)^{k+1} \geq 1+(k+1)x)$. Donc,
    \begin{align*}
        (1+x)^{k+1} &= (1+x)^{k} \times (1+x) \\
        (1+x)^{k+1} &\geq (1+kx) \times (1+x) \quad \xleftarrow[\text{Remarque que $1+x \geq 0$ car $x \geq -1$}]{\text{Par hypothèse d'induction}}\\
        (1+x)^{k+1} &\geq 1 + x + kx + kx^2\\
        (1+x)^{k+1} &\geq 1 + x(1+k) + kx^2
    \end{align*}
    Remarque que $x \geq -1$, donc que $x^2 \geq 0$. Puisque $x^2 \geq 0$, on a que $kx^2 \geq 0$. Par conséquent,
    $$1 + x(1+k) + kx^2 \geq 1 + x(1+k)$$
    Vu que, $$(1+x)^{k+1} \geq 1 + x(1+k) + kx^2$$ 
    Et $$1 + x(1+k) + kx^2 \geq 1 + x(1+k)$$
    On a que $$(1+x)^{k+1} \geq 1 + x(1+k)$$
    
    Puisque $P(k+1)$ est vrai pour tous $k \geq 2$, $x \geq -1$, l'énoncé est vrai.
    
    \newpage
    \item Faux
    
    Preuve par contradiction.
    
    Montrons que $$\neg \, \text{(l'énoncé)} \equiv \exists x \, \exists n \, \left((1+x)^n < 1 + nx\right)$$ est vrai dans le domaine $\{x \in \mathbb{R}\}$ et $\{n \in \mathbb{Z} \,|\, n \geq 2\}$.
    
    Prenons $n = 3$, $x = -5$. On voit que,
    
    Côté gauche:
    $$(1+x)^n = (1+(-5))^3 = (-4)^3 = -64$$
    
    Côté droit:
    $$1+nx = 1+(-5)(3) = 1 - 15 = -14$$
    Donc, $$(1+x)^n < 1 + nx$$ est faux lorsque $n = 3$, $x = -5$.
    
    Puisque la contradiction de l'énoncé est vrai, on peut conclure que l'énoncé est faux.
    
    
    
    


\end{enumerate}






\end{document} 

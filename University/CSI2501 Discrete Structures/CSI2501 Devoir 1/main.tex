
%% ==================================================================================================
%%
\documentclass[12pt]{book}
\usepackage{amsfonts}
\usepackage{amsmath}
\usepackage{amssymb}
\usepackage{graphicx}
\usepackage{hyperref}
\usepackage{float}
\usepackage{verbatim}
\usepackage{xlop} %% for multiplication https://tex.stackexchange.com/questions/11702/how-to-present-a-vertical-multiplication-addition
\usepackage{listings} %% to format generic computer code
\usepackage{lmodern} % for bold teletype font
\usepackage{minted} % colour Java code

\usepackage{tasks}
%\NewTasks[style=enumerate,counter-format=tsk[A].,label-width=3ex]{choice}[\item](4)

%% =======   set page margins    =======
\setlength{\textheight}{10in}
\setlength{\textwidth}{7.4in}
\setlength{\topmargin}{-0.75in}
\setlength{\oddsidemargin}{-0.5in}
\setlength{\evensidemargin}{-0.5in}
\setlength{\parskip}{0.15in}
\setlength{\parindent}{0in}

%%  for European long division
% https://tex.stackexchange.com/questions/432435/how-to-set-up-european-french-style-long-division-in-tex
\newcommand\frdiv[5]{%
    \[
    \renewcommand\arraystretch{1.5}
    \begin{array}{l| l}
    #1 & #2 \\
    \cline{2-2}
    #3 & #4 \\
    \cline{1-1}
    #5 & \\
    \end{array}
    \]
}

%%  for European long division


%% ==================================================================================================

\begin{document}

\newcommand{\reporttitle}{Devoir 1}
\newcommand{\reportauthorOne}{Kien Do}
\newcommand{\cidOne}{300163370}
\input{titlePage/titlepage.txt}



%% ==================================================================================================

%%%%%%%%%%%% PROBLEMS START HERE

\begin{enumerate}
    \item Faux
    
    La négation de l'énoncé est vraie.
    
    $$\neg \text{(l'énoncé)} \equiv \forall x \exists y \; (x^{y} \text{ est rationnel})$$
    
    Soit $x \in \mathbb{Q}$. Prenons $y = 1$. On voit que
    $x^1 = x$ peu importe la valeur de $x$. Puisque $x \in \mathbb{Q}$ et $x^1 = x$, il exist une valeur de $y$ telle que $x^y$ est toujours rationnel. Ceci montre que la négation de l'énoncé est vraie, donc que l'énoncé est faux.
    
    \item Vrai
    
    $$\forall x \exists y \; (3x + 2y \text{ est pair} \xrightarrow[]{} 2x + 3y \text{ est impair})$$
    
    Preuve directe.
    
    Soit $x$ et $y$ des entiers quelconques. $2y$ est toujours pair. 3x est seulement pair quand $x$ est pair parce que le produit des deux entiers impairs est impair. Donc, $3x + 2y$ est pair seulement si $x$ est pair. Puisque la valeur de $y$ ne détermine pas si $3x + 2y$ est pair ou non, prenons $y=1$, un entier impair arbitraire.\\
    
    Pour la deuxième partie de l'énoncé, on voit que $2x$ est toujours pair et $3y$ est impair parce que $y=1$ est un entier impair. Donc, $2x+3y$ est impair.\\
    
    Donc, l'énoncé est vrai.


\end{enumerate}






\end{document} 

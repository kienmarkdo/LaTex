
%% ==================================================================================================
%%
\documentclass[12pt]{book}
\usepackage{amsfonts}
\usepackage{amsmath}
\usepackage{amssymb}
\usepackage{graphicx}
\usepackage{hyperref}
\usepackage{float}
\usepackage{verbatim}
\usepackage{xlop} %% for multiplication https://tex.stackexchange.com/questions/11702/how-to-present-a-vertical-multiplication-addition
\usepackage{listings} %% to format generic computer code
\usepackage{lmodern} % for bold teletype font
\usepackage{minted} % colour Java code

\usepackage{tasks}
%\NewTasks[style=enumerate,counter-format=tsk[A].,label-width=3ex]{choice}[\item](4)

%% =======   set page margins    =======
\setlength{\textheight}{10in}
\setlength{\textwidth}{7.4in}
\setlength{\topmargin}{-0.75in}
\setlength{\oddsidemargin}{-0.5in}
\setlength{\evensidemargin}{-0.5in}
\setlength{\parskip}{0.15in}
\setlength{\parindent}{0in}

%%  for European long division
% https://tex.stackexchange.com/questions/432435/how-to-set-up-european-french-style-long-division-in-tex
\newcommand\frdiv[5]{%
    \[
    \renewcommand\arraystretch{1.5}
    \begin{array}{l| l}
    #1 & #2 \\
    \cline{2-2}
    #3 & #4 \\
    \cline{1-1}
    #5 & \\
    \end{array}
    \]
}

%%  for European long division


%% ==================================================================================================

\begin{document}

\newcommand{\reporttitle}{Assignment 1}
\newcommand{\reportauthorOne}{Kien Do}
\newcommand{\cidOne}{300163370}
\input{titlePage/titlepage.txt}



%% ==================================================================================================

%%%%%%%%%%%% PROBLEMS START HERE

\begin{enumerate}
    %% ============================   New Item   ============================
    \item \textbf{Answer}
    
    See scanned hand-written page.
    
    %% ============================   New Item   ============================
    \item \textbf{Answer}
    
    If $n = 2$ and $A[2] - A[1] > 0$, return the result of $A[2] - A[1]$. $\qquad \xleftarrow[]{O(1)}$
    
    Otherwise,
    
    Copy the elements of array A into new array B. $\qquad \xleftarrow[]{O(n)}$
    
    Sort array B using merge sort. $\qquad \xleftarrow[]{O(n\log(n))}$
    
    Create two variables $numOne$ and $numTwo$. Let variable $i$ be 0 and variable $j$ be 1. $\qquad \xleftarrow[]{O(1)}$
    
    Iterate through array B, incrementing $i$ and $j$ by 1 after each iteration. In each iteration, calculate $B[j] - B[i]$. If $B[j] - B[i] > 0$ and $B[j] - B[i] < minimum$,
    
    - reassign $minimum$ with $B[j] - B[i]$,
    
    - reassign $numOne$ with $B[i]$,
    
    - reassign $numTwo$ with $B[j]$.
    
    Stop the iteration when $j$ equals $n$. $\qquad \xleftarrow[]{O(n)}$
    
    Create variables $indexOne$ and $indexTwo$. 
    
    Go back to array A, iterate through it, then set $indexOne$ as the current index in the iteration if the current number in array A equals $numOne$ and set $indexTwo$ as the current index in the iteration if the current number in array A equals $numTwo$. Upon first assignment of $indexOne$ or $indexTwo$, do not reassign a new index to $indexOne$ or $indexTwo$ again. $\qquad \xleftarrow[]{O(n)}$
    
    Return $indexOne$ and $indexTwo$, where $indexOne$ and $indexTwo$ represent the indices of array A such that $A[indexTwo] - A[indexOne] > 0$ and $A[indexTwo] - A[indexOne]$ is the minimum.\\
    
    Time complexity:
    
    Recall that the time complexity of an algorithm is the sum of the time complexity of every operation. Ignoring the constant $O(1)$ time complexities, we have that,
    $$T(n) = O(n\log(n)) + O(n) + O(n)$$
    where $T(n)$ is the non-constant time complexity of the algorithm.
    
    We have that,
    \begin{align*}
        T(n) = n\log(n) + n + n &\leq n\log(n) + 2n\\
        &\leq n\log(n) + 2n\log(n)\\
        &\leq 3n\log(n)\\
        &= O(n\log(n))
    \end{align*}
    
    Therefore, the algorithm runs in $O(n\log(n))$
    
    
    
    
\end{enumerate}

\end{document} 

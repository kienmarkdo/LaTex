%%
%%
\documentclass[12pt]{book}
\usepackage{amsfonts}
\usepackage{amsmath}
\usepackage{amssymb}
\usepackage{graphicx}
\usepackage{hyperref}
\DeclareMathOperator\cis{cis} %defining cis as an function
\setlength{\textheight}{10in}
\setlength{\textwidth}{7.4in}
\setlength{\topmargin}{-0.75in}
\setlength{\oddsidemargin}{-0.5in}
\setlength{\evensidemargin}{-0.5in}
\setlength{\parskip}{0.15in}
\setlength{\parindent}{0in}

\begin{document}


\vspace{-1.0in}\begin{center}
\Large{MHF4UR : Pre-AP Advanced Functions }

\Large{Assignment \#4}


\end{center}

%\medskip

\vspace{0.015in}\hrulefill\ 

\textbf{Reference Declaration} %  Fill in your Reference Declarations in this section before your submit your assignment.

Complete the Reference Declaration section below in order for your assignment to be graded.

If you used any references beyond the course text and lectures (such as other texts, discussions with colleagues or online resources), indicate this information in the space below.  If you did not use any aids, state this in the space provided. 

Be sure to cite appropriate theorems throughout your work. You may use shorthand for well-known theorems like the FT (Factor Theorem), RRT (Rational Root Theorem), etc. 

Note: Your submitted work must be \textbf{your original work}. 

Family Name: Do\\%Family Name Here
First Name: Kien%First Name Here

Declared References: \\
Looked for help on Stack Exchange for question 1e\\ https://math.stackexchange.com/questions/3315/what-is-sqrti \\

Discussed with peers, asked my dad and used this link to help me with question 4a\\
https://www.youtube.com/watch?v=b2X7MHK\textunderscore3ac
%% Actual link without the underscore _
%% https://www.youtube.com/watch?v=b2X7MHK_3ac

% Type your references here.
% You can use as many lines as required.

\vspace{0.015in}\hrulefill\ 

\newpage

%%%%%%%%%%%% PROBLEMS START HERE

\begin{enumerate}

%% PROBLEM 1
\item Answer the following short problems.

\begin{enumerate}
%% 1a
\item Express the complex number $z=\frac{1+i}{2-4i}$ in the form $a+bi$ where $(a,b \in \mathbb{R})$\\

\textbf{Solution:}
\begingroup
\addtolength{\jot}{0.5em}
\begin{align}
    z &= \dfrac{1+i}{2-4i}\\
    z &= \dfrac{1+i}{2-4i} \times \dfrac{2+4i}{2+4i} \xleftarrow[]{\textbf{multiplying by conjugate}}\\
    z &= \dfrac{2+4i+2i+4i^2}{4-16i^2}\\
    z &= \dfrac{2+4i+2i-4}{4+16}\\
    z &= \dfrac{-2+6i}{20}\\
    z &= \dfrac{-1}{10} + \dfrac{3}{10}i
\end{align}\\
$\therefore z = \dfrac{-1}{10} + \dfrac{3}{10}i$
\endgroup
%% 1b
\item Express the complex number $z=1-3i$ in polar form. \\

\textbf{Solution:}\\
The polar form is $r\times \cis{\theta}$. Determine $r$ using $r = \sqrt{a^2 + b^2}$ where $a$ is the real number, 1, and $b$ is the coefficient of $i$, -3. Determine $\theta$ by taking $\tan^{-1} \left(\dfrac{b}{a}\right)$.\\
\textit{Note: All angles are expressed as radian.}\\

$r = \sqrt{1^2 + (-3)^2}$ \\
$r = \sqrt{10}$ \\
$\theta = \tan^{-1} \left(\dfrac{-3}{1}\right)$\\
$\theta \approx -1.25$\\

$\therefore 1 - 3i $ in polar form is $\sqrt{10} \cis(-1.25)$
%% 1c
\item Express the complex number $z=1-3i$ in a different polar form than in \textbf{part b}.\\

\textbf{Solution:}\\
To express the complex number $z=1-3i$ in a different polar form than in part b, I can add $2\pi$ to the angle -1.25.\\
$-1.25 + 2\pi \approx 5.033$\\

$\therefore \sqrt{10} \cis(-1.25) = \sqrt{10} \cis(5.033)$\\\\
%% 1d
\item Express the complex number $r=3\cis(\frac{3\pi}{4})$ in rectangular form.\\

\textbf{Solution:}\\
Determine $a$ and $b$ in the rectangular form $a + bi$.\\

$\cos \dfrac{3\pi}{4} = \dfrac{a}{3}$\\

$\sin \dfrac{3\pi}{4} = \dfrac{b}{3}$\\

Rearrange both equations, we have, \\
$a = 3 \cos \dfrac{3\pi}{4}$ \\
$a = 3 \times \left(-\dfrac{\sqrt{2}}{2}\right)$ \\
$a = - \dfrac{3\sqrt{2}}{2}$ \\

$b = 3 \sin \dfrac{3\pi}{4}$\\
$b = 3 \times \left(\dfrac{\sqrt{2}}{2}\right)$\\
$b =  \dfrac{3\sqrt{2}}{2}$

$\therefore $ the complex number $r$ in rectangular form is $- \dfrac{3\sqrt{2}}{2} + \dfrac{3\sqrt{2}i}{2}$

\item Evaluate $\sqrt{i}$.\\

\textbf{Solution:}\\
Let $\sqrt{i}$ be some complex number $a + bi$, we have,
\setcounter{equation}{0}
\begin{align}
    \sqrt{i} &= a +bi\\
    i &= (a+bi)^2 \\
    i &= a^2 + 2abi + b^2i^2 \\
    i &= a^2 - b^2 + 2abi
\end{align}
We can see that the entire right hand side of line (4) is essentially a complex number in the form $a + bi$ where $a = a^2 - b^2$ and $b = 2abi$. The left hand side on line (4) is also a complex number in the form $a+bi$ where $a = 0$ and $b=1$. Since both sides of the equation are complex numbers, the left hand side's real part and imaginary part must be the same as the right hand side's real part and the imaginary part, respectively. Therefore, I can say that $a^2-b^2 = 0$ and $2abi = 1$. Let's take a closer look at those two equations.
\begin{align}
    a^2 - b^2 &= 0 \\
    a^2 = b^2 
\end{align}
Now the second equation.
\begin{align}
    2ab &= 1 \\
    a &= \dfrac{1}{2b}
\end{align}
Sub in $a = \dfrac{1}{2b}$ from line (8) into $a^2 = b^2$ on line (6).
\begin{align}
    \left(\dfrac{1}{2b}\right)^2 &= b^2 \\
    \dfrac{1}{4b^2} &= b^2 \\
    \dfrac{1}{4} &= b^4 \\
    \sqrt[\leftroot{-1}\uproot{4}4]{\dfrac{1}{4}} &= \sqrt[\leftroot{-1}\uproot{3}4]{b^4} \\
    b &= \pm \dfrac{1}{\sqrt{2}}
\end{align}
On line (6), since $a^2 = b^2$ and $b = \pm \dfrac{1}{\sqrt{2}}$, it is also true that $a = \pm \dfrac{1}{\sqrt{2}}$. Now, I can sub in $a, b = \pm \dfrac{1}{\sqrt{2}}$ into the complex number $a + bi$. \\
$\therefore \sqrt{i} = \pm \dfrac{1}{\sqrt{2}} \pm \dfrac{1}{\sqrt{2}}i \quad \text{or} \quad \pm 1 \left(\dfrac{1}{\sqrt{2}} + \dfrac{1}{\sqrt{2}} i \right)$


\end{enumerate}

%% I would recommend sandwiching your solution to every problem between the kind of structure I have provided below re: initial \vspace, the Solution: heading and the ending \vspace.
%\vspace{0.3cm} 
%\textbf{Solution:}\\
% Your solution starts here.
%\vspace{0.3cm}

\newpage

%% PROBLEM 2
\item Recall $e^{i\theta} = \cos(\theta)+i\sin(\theta)$.
%% 2a
\begin{enumerate}
\item Show that $e^{-i\theta} = \cos(\theta) - i\sin(\theta)$\\

\textbf{Solution:}\\
\iffalse %% COMMENT=============================
Consider two numbers $a$ and $b$ where $a=b$. Now consider the equation $a \times m = b \times n$. Since $a=b$ and $a\times m = b\times n$, it is also true that $m = n$.

By the same token, we have $e^{i\theta} = \cos(\theta)+i\sin(\theta)$ where $a=e^{i\theta}$,  $b=\cos(\theta)+i\sin(\theta)$ and $e^{-i\theta} = \cos(\theta) - i\sin(\theta)$ where $m=e^{-i\theta}$, $n=\cos(\theta) - i\sin(\theta)$. I will demonstrate that $e^{-i\theta} = \cos(\theta) - i\sin(\theta)$ with the same idea. \\

First, take $(\cos(\theta)+i\sin(\theta)) \times (\cos(\theta) - i\sin(\theta))$.
\setcounter{equation}{0}
\begin{align}
    &= ( \cos(\theta)+i\sin(\theta) ) ( \cos(\theta) - i\sin(\theta) ) \\
    &= \cos^2\theta - (i\sin\theta)^2 \\
    &= \cos^2\theta + \sin^2\theta \\
    &= 1
\end{align}
Now, take $e^{i\theta} \times e^{-i\theta}$.
\begin{align}
    &= e^{i\theta} \times e^{-i\theta} \\
    &= e^0 \\
    &= 1
\end{align}
Therefore, $[ \cos(\theta)+i\sin(\theta) ] [ \cos(\theta) - i\sin(\theta) ] = e^{i\theta} \times e^{-i\theta} = 1$.\\

Since $e^{i\theta} = \cos(\theta)+i\sin(\theta)$ and $[ \cos(\theta)+i\sin(\theta) ] [ \cos(\theta) - i\sin(\theta) ] = e^{i\theta} \times e^{-i\theta}$, it is true that $e^{-i\theta} = \cos(\theta) - i\sin(\theta)$.
\fi%% COMMENT=============================
Consider Euler's formula $e^{i\theta} = \cos(\theta)+i\sin(\theta)$. This formula works for every angle $\theta$, therefore, I can sub in $\theta$ as $-\theta$. We have,
\setcounter{equation}{0}
\begin{align}
    e^{i\theta} &= \cos(\theta)+i\sin(\theta) \\
    e^{-i\theta} &= \cos(-\theta)+i\sin(-\theta) 
\end{align}
Since cosine functions are even functions, meaning $f(x) = f(-x)$ and sine functions are odd functions, meaning $f(x) = -f(-x)$, I can rewrite line (2) as follows.
\begin{align}
    e^{-i\theta} &= \cos(-\theta)+i\sin(-\theta) \\
    e^{-i\theta} &= \cos(\theta)-i\sin(\theta) 
\end{align}
$\therefore e^{-i\theta} = \cos(\theta) - i\sin(\theta)$


%% 2b
\item Use expressions for $e^{i\theta}$ and $e^{-i\theta}$ to determine expressions for $\sin(\theta)$ and $\cos(\theta)$ in terms of the exponential function.\\

\textbf{Solution:}\\
Recall $e^{i\theta} = \cos(\theta)+i\sin(\theta)$ and $e^{-i\theta} = \cos(\theta) - i\sin(\theta)$. I can use these two equations to express $\sin\theta$ and $\cos\theta$ in terms of exponential functions $e^{i\theta}$ and $e^{-i\theta}$.\\

\textbf{Determine an expression for $\cos\theta$ in terms of exponential functions}\\
First, I want to have $\cos\theta$ be the only term in the expression. By adding $e^{i\theta}$ and $e^{-i\theta}$ together, I can cancel out $i\sin\theta$ and $-i\sin\theta$.
\setcounter{equation}{0}
\begin{align}
    &= e^{i\theta} + e^{-i\theta} \\
    &= \cos\theta + i\sin\theta + \cos\theta - i\sin\theta \\
    &= 2\cos\theta
\end{align}
To get $\cos\theta$, I just need to divide $2\cos\theta$ on line (3) by 2.
\begin{align}
    &= \dfrac{2\cos\theta}{2} \\
    &= \cos\theta
\end{align}
$\therefore \cos\theta = \dfrac{e^{i\theta} + e^{-i\theta}}{2}$\\

\textbf{Determine an expression for $\sin\theta$ in terms of exponential functions}\\
Similarly, for $\sin\theta$, I need to cancel out $\cos\theta$. I can see that subtracting $e^{i\theta}$ and $e^{-i\theta}$, I will get the desired result.
\begin{align}
    &= e^{i\theta} - e^{-i\theta} \\
    &= \cos\theta + i\sin\theta - \cos\theta + i\sin\theta \\
    &= 2i\sin\theta\\
    &= \dfrac{2\sin\theta}{2i} \\
    &= \sin\theta
\end{align}
$\therefore \sin\theta = \dfrac{e^{i\theta} - e^{-i\theta}}{2i}$

%% 2d
\item Use these definitions to prove that the identity $\sin^2(\theta) + \cos^2(\theta) = 1$ still holds for these new definitions of the sine and cosine functions.\\

\textbf{Solution:}\\
Sub in $\sin\theta =\dfrac{e^{i\theta} - e^{-i\theta}}{2i}$ and $ \cos\theta = \dfrac{e^{i\theta} + e^{-i\theta}}{2}$ into $\sin^2(\theta) + \cos^2(\theta)$ to see if it still equals 1.
\setcounter{equation}{0}
\begin{align}
    &= \sin^2(\theta) + \cos^2(\theta) \\
    &= \left(\dfrac{e^{i\theta} - e^{-i\theta}}{2i}\right)^2 + \left(\dfrac{e^{i\theta} + e^{-i\theta}}{2} \right)^2 \\
    &= \left(\dfrac{\cos\theta + i\sin\theta - \cos\theta + i\sin\theta}{2i}\right)^2 + \left(\dfrac{\cos\theta + i\sin\theta + \cos\theta - i\sin\theta}{2} \right)^2  \\
    &= \left(\dfrac{2i\sin\theta}{2i}\right)^2 + \left(\dfrac{2\cos\theta}{2} \right)^2  \\
    &= \sin^2\theta + \cos^2\theta \\
    &= 1
\end{align}
$\therefore \left(\dfrac{e^{i\theta} - e^{-i\theta}}{2i}\right)^2 + \left(\dfrac{e^{i\theta} + e^{-i\theta}}{2} \right)^2 = 1$







\end{enumerate}

\newpage

%% PROBLEM 3
\item Prove that there does not exist a complex number $z$ such that $|z| - z = i$.\\

\vspace{0.2cm}
(Hint: Proof by contradiction - assume that such a $z$ exists and show that this assumption leads to a contradiction)\\

\textbf{Solution:}\\
Assuming that such $z$ exists, sub in $z = a+bi$ and $|z| = \sqrt{a^2+b^2}$ into $|z| - z = i$. We have,\\
\setcounter{equation}{0}
\begin{align}
    i &=|z| - z \\
    i &= \sqrt{a^2+b^2} - (a+bi) \\
    i &= \sqrt{a^2+b^2} - a - bi 
\end{align}
Looking at line (3), we can see the both the left hand side and the right hand side are complex numbers. The left hand side is a complex number $a +bi$ where $a = 0, b = 1$ and the right hand side is a complex number where the real part is $\sqrt{a^2+b^2} - a$ and the imaginary part is $-b$. In order for the right hand side to equal to the left hand side, $\sqrt{a^2+b^2} - a$ \textbf{must be} zero and $b$ \textbf{must be} -1. Let's examine the real part of the right hand side of the equation on line (3).\\

We know that $\sqrt{a^2+b^2} - a= 0$ and $b = -1$. Sub in $b = -1$ into $\sqrt{a^2+b^2} - a = 0$. We have,
\begin{align}
    \sqrt{a^2+(-1)^2} - a &= 0 \\
    \sqrt{a^2+1} - a &= 0 
\end{align}
For line (5) to be true, $\sqrt{a^2+1}$ must equal $a$. However, we can see that the statement on line (5) is false as $\sqrt{a^2+1}$ is always larger than $a$. If $\sqrt{a^2} = a$ (if we ignore the negative value), then $\sqrt{a^2+1} > a$. Therefore $\sqrt{a^2+1}-a > 0$. But, the real part of $\sqrt{a^2+b^2} - a$ needs to be zero! We have run into a contradiction.\\

$\therefore \qquad \textbf{there does not exist a complex number such that} \; |z|-z=i$

\newpage

%% PROBLEM 4
\item Recall de Moivre's Theorem.

\begin{enumerate}
% 4a
\item Use de Moivre's Theorem to show that $\sin(5\theta) = 16\sin^5(\theta) - 20\sin^3(\theta) + 5\sin(\theta)$.\\

\textbf{Solution:}\\
De Moivre's theorem states that $r^n[\cos(n\theta)+i\sin(n\theta)] = [r(\cos\theta+i\sin\theta)]^n$. In this solution, I will let $r=1$, $n=5$.
\setcounter{equation}{0}
\begin{align}
    &= 1^5 (\cos(5\theta) + i\sin(5\theta)) \\
    &= \cos(5\theta) + i\sin(5\theta) \\
    &= (\cos\theta + i\sin\theta)^5 \\
    \begin{split}
        &= \cos^5\theta + 5\cos^4\theta(i\sin\theta) + 10\cos^3\theta(i\sin\theta)^2 + 10\cos^2\theta(i\sin\theta)^3 + \\
        & \quad  5\cos\theta(i\sin\theta)^4 + (i\sin)^5\theta\\
    \end{split}\\
    &= \cos^5\theta - 10 \cos^3\theta\sin^2\theta + 5\cos\theta\sin^4\theta + i[5\cos^4\theta\sin\theta - 10\cos^2\sin^3\theta + \sin^5\theta]\\
    \begin{split}
        &= \cos^5\theta - 10\cos^3\theta(1-\cos^2\theta) + 5\cos\theta(1-\cos^2\theta)^2 \\
        & \quad +i[5\sin\theta(1-\sin^2\theta)^2 - 10\sin^3\theta(1-\sin^2\theta) + \sin^5\theta]
    \end{split}\\
    \begin{split}
        &= \cos^5\theta - 10\cos^3\theta + 10\cos^5\theta + 5\cos\theta - 10\cos^3\theta + 5\cos^5\theta \\
        & \quad i[5\sin\theta - 10\sin^3\theta + 5\sin^5\theta - 10\sin^3\theta + 10\sin^5\theta + \sin^5\theta]\\
    \end{split}\\
    &= 16\cos^5\theta - 20\cos^3\theta + 5\cos\theta + i(16\sin^5\theta - 20\sin^3\theta + 5\sin\theta)
\end{align}
\textit{Explanation of work done above:}\\
- Since the question asked me to show that $sin(5\theta) = \sin(5\theta) = 16\sin^5(\theta) - 20\sin^3(\theta) + 5\sin(\theta)$, I started line (1) with $\cis(5\theta)$ (but written out fully) so that the argument of cis is the same as sine.\\
- From line (2) to line (3), I used De Moivre's theorem to rewrite the equation.\\
- From line (3) to line (4), I expanded the equation using Pascal's Triangle.\\
- From line (4) to line (5), I grouped all of the "real" parts together and the "imaginary" parts together. The real parts have values of $i^n$ where $n$ is a positive even integer. Similarly, I the imaginary parts are grouped when the exponent of $i$ is a positive odd integer. \\
- On line (6), I replaced all $\sin^2\theta$ in the real part in terms of cosine and I replaced all the $\cos^2\theta$ in terms of sine. This will make sense in line (9) \\
- On line (7), I expanded the equation after subbing in new values from line (6). \\
- Line (8) is line (7) but simplified. If we take a closer look at line (8), we can see that it resembles a $\cis$ function. The cosine part, or, the real part, is $16\cos^5\theta - 20\cos^3\theta + 5\cos\theta$ and the $i$sine part, or, the imaginary part, is $i(16\sin^5\theta - 20\sin^3\theta + 5\sin\theta)$. \\

Notice, line (1) equals line (8). That means the corresponding real part and imaginary part must equal. Therefore, I can say,\\
$$\cos(5\theta) = 16\cos^5\theta - 20\cos^3\theta + 5\cos\theta$$
$$i\sin(5\theta) = i(16\sin^5\theta - 20\sin^3\theta + 5\sin\theta)$$
$$\sin(5\theta) = 16\sin^5\theta - 20\sin^3\theta + 5\sin\theta$$
Since the question asked to show that $\sin(5\theta) = 16\sin^5\theta - 20\sin^3\theta + 5\sin\theta$, we only care about the third line. Therefore, I have shown that $\sin(5\theta) = 16\sin^5\theta - 20\sin^3\theta + 5\sin\theta$.\\



% 4b
\item Given this, show that the polynomial equation $16x^5 - 20x^3 + 5x - 1 = 0$ has roots $x=1, \sin(\frac{\pi}{10}), \sin(\frac{9\pi}{10}), \sin(\frac{13\pi}{10})$ and $\sin(\frac{17\pi}{10})$. \\

\textbf{Solution:}\\
Sub in $x=1, \sin(\frac{\pi}{10}), \sin(\frac{9\pi}{10}), \sin(\frac{13\pi}{10})$ and $\sin(\frac{17\pi}{10})$ into the polynomial equation to see if the equation equals 0 or not.\\

Sub in $x=1$
\setcounter{equation}{0}
\begin{align}
    &= 16x^5 - 20x^3 + 5x - 1 \\
    &= 16(1)^5 - 20(1)^3 + 5(1) - 1 \\
    &= 0
\end{align}
\iffalse %%============================= comment ============================
Sub in $x=\sin(\frac{\pi}{10})$
\begin{align}
    0 &= 16\left(\sin\left(\dfrac{\pi}{10}\right)\right)^5 - 20\left(\sin\left(\dfrac{\pi}{10}\right)\right)^3 + 5\left(\sin\left(\dfrac{\pi}{10}\right)\right) - 1
\end{align}
Sub in $x=\sin(\frac{9\pi}{10})$
\begin{align}
    0 &= 16\left(\sin\left(\dfrac{9\pi}{10}\right)\right)^5 - 20\left(\sin\left(\dfrac{9\pi}{10}\right)\right)^3 + 5\left(\sin\left(\dfrac{9\pi}{10}\right)\right) - 1
\end{align}
Sub in $x=\sin(\frac{13\pi}{10})$
\begin{align}
    0 &= 16\left(\sin\left(\dfrac{13\pi}{10}\right)\right)^5 - 20\left(\sin\left(\dfrac{13\pi}{10}\right)\right)^3 + 5\left(\sin\left(\dfrac{13\pi}{10}\right)\right) - 1 
\end{align}
Sub in $x=\sin(\frac{17\pi}{10})$
\begin{align}
    0 &= 16\left(\sin\left(\dfrac{17\pi}{10}\right)\right)^5 - 20\left(\sin\left(\dfrac{17\pi}{10}\right)\right)^3 + 5\left(\sin\left(\dfrac{17\pi}{10}\right)\right) - 1 
\end{align}\\
\fi  %%============================= comment ============================
This polynomial function looks similar to the one from question a, so, I will sub in $x = \sin\theta$.
\begin{align}
    0 &= 16x^5 - 20x^3 + 5x - 1 \\
    0 &= 16\sin^5\theta - 20\sin^3\theta + 5\sin\theta - 1 \\
    1 &= 16\sin^5\theta - 20\sin^3\theta + 5\sin\theta \\
    1 &= \sin(5\theta) \xleftarrow[]{\textbf{sub in from a}}
\end{align}
Now, I can sub in all the angles/roots from this question into the equation on line (7).\\
Sub in $\theta = \dfrac{\pi}{10}$
\begin{align}
    1 &= \sin\left(5\dfrac{\pi}{10}\right) \\
    1 &= 1
\end{align}
Sub in $\theta = \dfrac{9\pi}{10}$
\begin{align}
    1 &= \sin\left(5\dfrac{9\pi}{10}\right) \\
    1 &= 1
\end{align}
Sub in $\theta = \dfrac{13\pi}{10}$
\begin{align}
    1 &= \sin\left(5\dfrac{13\pi}{10}\right) \\
    1 &= 1
\end{align}
Sub in $\theta = \dfrac{17\pi}{10}$
\begin{align}
    1 &= \sin\left(5\dfrac{17\pi}{10}\right) \\
    1 &= 1
\end{align}

$\therefore \quad$ the polynomial equation $16x^5 - 20x^3 + 5x - 1 = 0$ has roots x=1, $\sin(\frac{\pi}{10}), \sin(\frac{9\pi}{10}), \sin(\frac{13\pi}{10})$  and $\sin(\frac{17\pi}{10})$.



% 4c
\item By equating coefficients, determine the values of $b$ and $c$ for which $16x^4 + 16x^3 -4x^2 - 4x + 1 = (4x^2 + bx + c)^2$ and then explain why $16x^4 + 16x^3 -4x^2 - 4x + 1 = 0$ has two roots of multiplicity 2. \\

\textbf{Solution:}\\
Determine $b$ and $c$ then determine the roots of $4x^2 + bx + c$. There should be two real roots and since $4x^2 + bx + c$ is squared, each of those roots have multiplicity 2. \\

Determine the values of $b$ and $c$. Expand $(4x^2 + bx + c)^2$ using the polynomial identity $(a+b)^2 = a^2+2ab+b^2$ where $a=4x^2$, $b= bx+c$.
\setcounter{equation}{0}
\begin{align}
    &= (4x^2 + bx + c)^2 \\
    &= 16x^4 +2(4x^2)(bx+c)+(bx+c)^2 \\
    &= 16x^4 +8x^2(bx+c)+(bx+c)^2 \xleftarrow[(bx+c)^2]{\textbf{polynomial identity on}} \\
    &= 16x^4 +8bx^3+8cx^2+b^2x^2+2bxc+c^2 \\
    &= 16x^4 +8bx^3+ (8c+b^2)x^2+2bcx+c^2
\end{align}
Since $16x^4 +8bx^3+ (8c+b^2)x^2+2bcx+c^2 = (4x^2 + bx + c)^2$ and $16x^4 + 16x^3 -4x^2 - 4x + 1 = (4x^2 + bx + c)^2$, it is also true that $16x^4 +8bx^3+ (8c+b^2)x^2+2bcx+c^2 = 16x^4 + 16x^3 -4x^2 - 4x + 1$.\\

This means that the coefficients of $x^4$, $x^3$, $x^2$, $x^4$ from the original equation are all equal to the coefficients of $x$ in the equation $16x^4 +8bx^3+ (8c+b^2)x^2+2bcx+c^2$ on line (5). Just looking at the two equations, I can already determine that the coefficient of $x^4$ on both equations are 16, and $c^2=1$.\\

Since $c^2 = 1$, $c^2 = \pm 1$. Sub in $c=1$ and $c=-1$ to determine the values of $b$ then sub in those values in $16x^4 +8bx^3+ (8c+b^2)x^2+2bcx+c^2$ to see which pair satisfies the coefficient of $x^3,x^2$ and $x$. Sub in $c = 1$ where necessary to determine $b$, then check to see if it is valid.
$8b = 16$ \\
$8c+b^2 = -4$ \\
$2bc = -4$ \\
Sub in $c=1$ in $2bc = -4$ to determine $b$
\begin{align}
    2bc = -4 \\
    2(1)b &= -4 \\
    b &= -2
\end{align}
Sub in $b=-2, c=1$ into the equation on line (5) to see if it equals to original equation given in the question.
\begin{align}
    &= 16x^4 +8bx^3+ (8c+b^2)x^2+2bcx+c^2 \\
    &= 16x^4 +8(-2)x^3+ (8(1)+(-2)^2)x^2 +2(-2)(1)x+(1)^2 \\
    &= 16x^4 - 16x^3 + 4x^2 - 4x+ 1
\end{align}
The equation on line (11) does not equal to the original equation $16x^4 + 16x^3 -4x^2 - 4x + 1$. Determine a new pair of values for $b$ and $c$ by subbing in $c=-1$ to determine $b$, then sub those two values into the equation on line (5), again, to check if it equals the original value given in the question.\\
Sub in $c=-1$ in $2bc = -4$ to determine $b$
\begin{align}
    2bc = -4 \\
    2(-1)b &= -4 \\
    b &= 2
\end{align}
Sub in $b=2, c=-1$ into the equation on line (5) to see if it equals to original equation given in the question.
\begin{align}
    &= 16x^4 +8bx^3+ (8c+b^2)x^2+2bcx+c^2 \\
    &= 16x^4 +8(2)x^3+ (8(-1)+(2)^2)x^2 +2(2)(-1)x+(-1)^2 \\
    &= 16x^4 + 16x^3 - 4x^2 - 4x+ 1
\end{align}
Now that the values of $b$ and $c$ are determined to be 2 and -1, respectively, determine the number of real roots by subbing in those values into the given equation $(4x^2 + bx + c)^2$. We have,
$$(4x^2 + bx + c)^2 = (4x^2 + 2x - 1)^2$$
Determine roots using the quadratic equation $$\Delta = b^2-4ac, \quad x = \dfrac{-b\pm \sqrt{\Delta}}{2a}$$
\begin{align}
    \Delta &= 2^2 - 4(4)(-1) \\
    \Delta &= 20 \\
    x_1 &= \dfrac{-2 + \sqrt{20}}{8} \\
    x_2 &= \dfrac{-2 - \sqrt{20}}{8}
\end{align}
We can see that there are two real roots, $x_1 = \dfrac{-2 + \sqrt{20}}{8},    x_2 = \dfrac{-2 - \sqrt{20}}{8}$. However, since the equation $(4x^2 + 2x - 1)^2$ is squared, each root would be of multiplicity 2. A visual way to represent this is by factoring out $(4x^2 + 2x - 1)^2$ to $\left(x-\dfrac{-2 + \sqrt{20}}{8}\right)^2\left(x+\dfrac{-2 - \sqrt{20}}{8}\right)^2$. This is why $16x^4 - 16x^3 + 4x^2 - 4x+ 1$ has two real roots of multiplicity 2. \\


% 4d
\item Use \textbf{part b} to show that the equation $16x^4 + 16x^3 - 4x^2 -4x + 1 = 0$ has roots $x=\sin(\frac{\pi}{10}), \sin(\frac{9\pi}{10}), \sin(\frac{13\pi}{10})$ and $\sin(\frac{17\pi}{10})$. Does this contradict \textbf{part c} which asserts that the equation has two roots of multiplicity 2?\\

\textbf{Solution:}\\
Take a look at the equation in question 4b, it is very similar to the equation in this question. Through long division, I took $16x^5 - 20x^3 + 5x - 1$ and divided it by $16x^4 + 16x^3 - 4x^2 - 4x + 1$ and that equalled $(x-1)$.\\

I can rewrite the equation in question 4b so that it looks like the equation in this question by factoring out $(x-1)$.
\setcounter{equation}{0}
\begin{align}
    0 &= 16x^5 - 20x^3 + 5x - 1 \\
    0 &= (x-1)(16x^4 + 16x^3 - 4x^2 - 4x + 1)
\end{align}
Taking a look at line (2) above, it is still the same equation as from question 4c which we know has the following roots $x=1, \sin(\frac{\pi}{10}), \sin(\frac{9\pi}{10}), \sin(\frac{13\pi}{10})$  and $\sin(\frac{17\pi}{10})$. \\

However, after factoring out (x-1) from the equation from question 4b, we have the factor $(16x^4 + 16x^3 - 4x^2 - 4x + 1)$ which is the same as the equation from 4c, and from 4c, we already know that the roots for this equation are $x=\sin(\frac{\pi}{10}), \sin(\frac{9\pi}{10}), \sin(\frac{13\pi}{10})$  and $\sin(\frac{17\pi}{10})$.

That means, all of the roots from 4c are the same as 4b except for x=1 because that has been factored out. So,\\

$16x^4 + 16x^3 - 4x^2 - 4x + 1$ has roots $x=\sin(\frac{\pi}{10}), \sin(\frac{9\pi}{10}), \sin(\frac{13\pi}{10})$ and $\sin(\frac{17\pi}{10})$. \\

Note that $\sin\dfrac{\pi}{10} = \sin\dfrac{9\pi}{10}$ because they are related acute angles. Likewise, $\sin\dfrac{13\pi}{10} = \sin\dfrac{17\pi}{10}$. That means, the equation from part C actually only has two real roots, but since there are two equivalent roots for each real root, each real root is of multiplicity 2.\\


Therefore, I can come to the conclusion that this does not contradict part C because $x_1 = \sin\dfrac{\pi}{10} = \sin\dfrac{9\pi}{10}$ and $x_2 = \sin\dfrac{13\pi}{10} = \sin\dfrac{17\pi}{10}$.\\



\iffalse %% ================================= comment =======================
Sub in $x=\sin(\frac{\pi}{10}), \sin(\frac{9\pi}{10}), \sin(\frac{13\pi}{10})$ and $\sin(\frac{17\pi}{10})$ into $16x^4 + 16x^3 - 4x^2 -4x + 1 = 0$ to see if the the equation equals 0 or not.\\

Sub in $\sin(\frac{\pi}{10})$
\setcounter{equation}{0}
\begin{align}
    &= 16x^4 + 16x^3 - 4x^2 -4x + 1 \\
    &= 16\left(\sin\left(\dfrac{\pi}{10}\right) \right)^4 + 16\left(\sin\left(\dfrac{\pi}{10}\right) \right)^3 - 4\left(\sin\left(\dfrac{\pi}{10}\right) \right)^2 -4\left(\sin\left(\dfrac{\pi}{10}\right) \right) + 1 \\
    &= 0
\end{align}
Sub in $\sin(\frac{9\pi}{10})$
\begin{align}
    0 &= 16\left(\sin\left(\dfrac{9\pi}{10}\right) \right)^4 + 16\left(\sin\left(\dfrac{9\pi}{10}\right) \right)^3 - 4\left(\sin\left(\dfrac{9\pi}{10}\right) \right)^2 -4\left(\sin\left(\dfrac{9\pi}{10}\right) \right) + 1
\end{align}
Sub in $\sin(\frac{13\pi}{10})$
\begin{align}
    0 &= 16\left(\sin\left(\dfrac{13\pi}{10}\right) \right)^4 + 16\left(\sin\left(\dfrac{13\pi}{10}\right) \right)^3 - 4\left(\sin\left(\dfrac{13\pi}{10}\right) \right)^2 -4\left(\sin\left(\dfrac{13\pi}{10}\right) \right) + 1
\end{align}
Sub in $\sin(\frac{17\pi}{10})$
\begin{align}
    0 &= 16\left(\sin\left(\dfrac{17\pi}{10}\right) \right)^4 + 16\left(\sin\left(\dfrac{17\pi}{10}\right) \right)^3 - 4\left(\sin\left(\dfrac{17\pi}{10}\right) \right)^2 -4\left(\sin\left(\dfrac{17\pi}{10}\right) \right) + 1
\end{align}
Before I answer the question, note that $\sin\dfrac{\pi}{10} = \sin\dfrac{9\pi}{10}$ because they are related acute angles. Likewise, $\sin\dfrac{13\pi}{10} = \sin\dfrac{17\pi}{10}$. Now, to answer the question - no, this does not contradict part C because $x_1 = \sin\dfrac{\pi}{10} = \sin\dfrac{9\pi}{10}$ and $x_2 = \sin\dfrac{13\pi}{10} = \sin\dfrac{17\pi}{10}$.
\fi %% ================================= comment =======================

% 4e
\item Given your findings, determine exact values for $\sin(\frac{\pi}{10})$ and $\sin(\frac{3\pi}{10})$\\

\textbf{Solution:}\\

From part C and part D, note that $$\sin\left(\dfrac{\pi}{10}\right) = \left(\dfrac{-2 + \sqrt{20}}{8}\right)$$ and $$\sin\left(\dfrac{17\pi}{10}\right) = \left(\dfrac{-2 - \sqrt{20}}{8}\right)$$

Recall the unit circle from trigonometry. The sine function has positive values in quadrants 1 and 2 and negative values in quadrants 3 and 4 and we know that a circle's entire angle is $2\pi$.\\

We do not need to determine the exact value of $\sin\left(\dfrac{\pi}{10}\right)$ because we already know from part C and part D that $\sin\left(\dfrac{\pi}{10}\right) = \left(\dfrac{-2 + \sqrt{20}}{8}\right)$, therefore, nothing needs to be done.\\

However, we do not yet know the exact value of $\sin\left(\dfrac{17\pi}{10}\right)$. Note what I recalled earlier from trigonometry, I see that $\sin\left(\dfrac{17\pi}{10}\right)$ is in quadrant 4. That means the absolute value of $\sin\left(\dfrac{17\pi}{10}\right)$ is the same as its related angle in quadrant 1, which is $\sin\left(\dfrac{3\pi}{10}\right)$.\\

Therefore,
\setcounter{equation}{0}
\begin{align}
    \sin\left(\dfrac{17\pi}{10}\right) &= \sin\left(\dfrac{3\pi}{10}\right) \times (-1) \\
    \sin\left(\dfrac{17\pi}{10}\right) &= \left(\dfrac{-2 + \sqrt{20}}{8}\right) \times (-1) \\
    \sin\left(\dfrac{17\pi}{10}\right) &= \left(\dfrac{2 + \sqrt{20}}{8}\right) \\
    \sin\left(\dfrac{17\pi}{10}\right) &= \left(\dfrac{2 + \sqrt{5\times4}}{8}\right) \\
    \sin\left(\dfrac{17\pi}{10}\right) &= \left(\dfrac{2 + 2\sqrt{5}}{8}\right) \\
    \sin\left(\dfrac{17\pi}{10}\right) &= \left(\dfrac{2(1 + \sqrt{5})}{8}\right) \\
    \sin\left(\dfrac{17\pi}{10}\right) &= \left(\dfrac{1 + \sqrt{5}}{4}\right)
\end{align}

Therefore, given my findings from part C and part D, the exact values for $\sin(\frac{\pi}{10})$ and $\sin(\frac{3\pi}{10})$ are,
$$\sin\left(\dfrac{\pi}{10}\right) = \left(\dfrac{-1 + \sqrt{5}}{4}\right)$$ and $$\sin\left(\dfrac{3\pi}{10}\right) = \left(\dfrac{1 + \sqrt{5}}{4}\right)$$

\end{enumerate}


\newpage


\end{enumerate}
\end{document} 
